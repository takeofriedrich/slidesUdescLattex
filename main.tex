\documentclass[aspectratio=169]{beamer}

% Pacote de estilo da UDESC
\usepackage{style/udesc}
\usepackage{listings}
\setbeamertemplate{itemize items}[circle]
\usepackage[abnt-emphasize=bf,abnt-and-type=e,alf]{abntex2cite}%Citações ABNT
\usepackage{amsmath}

% Incluir arquivos da pasta figuras
\graphicspath{{./figuras/}}

\setbeamertemplate{frametitle continuation}{}

% Pacote de texto aleatório
\usepackage{lipsum}

\lstset{ 
  basicstyle=\footnotesize,        % the size of the fonts that are used for the code
  breakatwhitespace=false,         % sets if automatic breaks should only happen at whitespace
  breaklines=true,                 % sets automatic line breaking
  captionpos=b,                    % sets the caption-position to bottom
  deletekeywords={...},            % if you want to delete keywords from the given language
  escapeinside={\%*}{*)},          % if you want to add LaTeX within your code
  extendedchars=true,              % lets you use non-ASCII characters; for 8-bits encodings only, does not work with UTF-8
  firstnumber=0,                % start line enumeration with line 1000
  frame=single,	                   % adds a frame around the code
  keepspaces=true,                 % keeps spaces in text, useful for keeping indentation of code (possibly needs columns=flexible)
  language=Java,                 % the language of the code
  morekeywords={*,...},            % if you want to add more keywords to the set
  numbers=none,                    % where to put the line-numbers; possible values are (none, left, right)
  numbersep=0,                   % how far the line-numbers are from the code
   rulecolor=\color{black},         % if not set, the frame-color may be changed on line-breaks within not-black text (e.g. comments (green here))
  showspaces=false,                % show spaces everywhere adding particular underscores; it overrides 'showstringspaces'
  showstringspaces=false,          % underline spaces within strings only
  showtabs=false,                  % show tabs within strings adding particular underscores
  stepnumber=2,                    % the step between two line-numbers. If it's 1, each line will be numbered
  tabsize=1,	                   % sets default tabsize to 2 
  basicstyle=\fontsize{7}{8}\selectfont\ttstyle,
  keywordstyle=\color{blue},
  commentstyle=\ttfamily\small\color{gray},
  stringstyle=\color{orange},
}


% Início do documento
\begin{document}

%%
%%	Incluir \capa para os slides
%% 
\titulo{Titulo da Apresentação}
\subtitulo{Subtitulo}
\newcommand{\autor}{Vinicius Takeo Friedrich Kuwaki}
\newcommand{\github}{github.com/takeofriedrich}
\newcommand{\email}{vtkwki@gmail.com}
\newcommand{\website}{}
\frase{Frase do dia}
\universidade{Universidade do Estado de Santa Catarina}
\capa

    \AtBeginSection[]{
    \begin{frame}<beamer>
        \frametitle{Seções}
        \tableofcontents[currentsection]
    \end{frame}}


\section{Tutorial}

\subsection{Slide Simples}

\begin{frame}{Introdução}
\lipsum[75]
\end{frame}


\subsection{Tópicos}
\subsubsection{Estrutura}

\begin{frame}{Estrutura de Tópicos}

    \begin{itemize}
        \item Tópico 1.
            \begin{itemize}
                \item Tópico 1.1
                    \begin{itemize}
                        \item Tópico 1.1.1
                        \item Tópico 1.1.2 
                    \end{itemize}
            \end{itemize}
    \end{itemize}
    
\end{frame}

\subsubsection{Fadding}

\begin{frame}{Fadding de Tópicos}
    \begin{itemize}
        \item Tópico 1 \pause
        \item Tópico 2 \pause
        \item Tópico 3 
    \end{itemize}
\end{frame}


%%
%%	Remover o número de páginas do slide com [plain]
%%

\subsection{Tipos de Slides}
\subsubsection{Texto sem número de páginas}

\begin{frame}[plain]{Sem número de páginas}
\lipsum[1]
\end{frame}


\subsubsection{Texto com imagem ao lado}
%%
%%	Você pode usar mais de uma coluna! Inclusão automatizada de imagens com \imagem{arquivo}{label}{legenda}
%%
\begin{frame}{Texto em duas colunas}
  \begin{columns}[c]
    \begin{column}{.5\textwidth}

	Exemplo de imagem

    \end{column}
    \begin{column}{.5\textwidth}

	\imagem{figuras/golden.jpg}{Dog}{Imagem de um Golden}

    \end{column}
  \end{columns}
\end{frame}

\begin{frame}{Apenas Imagem}

    \imagem{figuras/golden.jpg}{Dog2}{Imagem de um Golden}
    
\end{frame}

\subsubsection{Caixas de Texto}

%%
%%	Caixas de texto
%%
\begin{frame}{Caixas de Texto}

\begin{block}{Bloco Normal}
Conteúdo do bloco normal.
\end{block}

\begin{alertblock}{Bloco de Alerta}
Conteúdo do bloco de alerta.
\end{alertblock}

\begin{exampleblock}{Bloco de Exemplo}
Conteúdo do bloco de exemplo
\end{exampleblock}

\end{frame}

\subsubsection{Equações}

%%
%%	Equações
%%
\begin{frame}{Equações}

Observe a Equação \ref{eq:sum}.

% Iniciar ambiente equation
\begin{equation}\label{eq:sum}
\sum_{n=1}^\infty \frac{1}{n^2} = \lim_{n \to \infty} \left( \frac{1}{1^2} + \frac{1}{2^2} + \cdots + \frac{1}{n^2} \right) = \frac{\pi^2}{6}
\end{equation}

Observe as Equações \ref{eq:bhaskara}.

% Iniciar ambiente split dentro de equation para várias linhas
\begin{equation}\label{eq:bhaskara}
\begin{split}
x_1 = \frac{-b + \sqrt{b^2 - 4ac}}{2a} \\
x_2 = \frac{-b - \sqrt{b^2 - 4ac}}{2a}
\end{split}
\end{equation}

\end{frame}

\subsubsection{Código-Fonte}

\defverbatim[colored]\lstI{
\begin{lstlisting}[language=Java,basicstyle=\ttfamily,keywordstyle=\color{blue}]
    class Main{
    
        public static void main(String[] args){
            
            System.out.println("Hello World");
        
        }
    
    }
\end{lstlisting}
}

\begin{frame}{Código Fonte}{Java}
\lstI
\end{frame}

\begin{frame}{Titulo do slide}
    Exemplo de citação: \cite{github}
\end{frame}

\begin{frame}[allowframebreaks]
        \frametitle{Referencias}
        \bibliographystyle{abntex2-alf}
        \bibliography{referencias.bib}
        
\end{frame}

\contato{Contato: \\
\autor{} \\
\email{} \\
\github{} \\
\website{}
}
\capadetras{Obrigado!}

\end{document}
